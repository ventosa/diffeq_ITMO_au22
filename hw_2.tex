\begin{task}
\textbf{Скрудж МакДак} хочет создать трастовый фонд для своего племянника Дона. У него есть надежные инвестиции, которые обеспечивают постоянную процентную ставку $I$, измеряемую в единицах (годы$)^{−1}$ (таким образом, $I = 0.05$ означает 5 процентов в год), и он предлагает раздать деньги своему распутному племяннику по постоянная ставка $q$ долларов в год.
\begin{itemize}
    \item Если бы дядя Скрудж хотел финансировать траст, чтобы обеспечить его племянника с 1000 долларов в месяц на неограниченный срок, сохраняя при этом постоянный баланс в фонде, сколько он должен инвестировать?
    \item Но на самом деле дядя Скрудж хочет научить своего племянника уверенности в себе, и поэтому план на то, чтобы трастовый фонд полностью исчерпал деньги ровно через двадцать лет. Если он хочет давать своему племяннику 1000 долларов в месяц, сколько он должен финансировать в траст вначале? Дайте ответ с точностью до копейки (на чем настаивал Скрудж).
\end{itemize}

\end{task}

\begin{task}
Почти весь \textbf{радон}, существующий на планете в текущий момент, образовался в течение последней недели или около того в результате цепочки радиоактивных распадов, начавшейся в основном из урана. Рассмотрим упрощённую модель, что уран был частью планеты с момента ее образования. Этот каскад распадающихся элементов встречается довольно часто. 
Определенный изотоп Стартиум, обозначенный $St$, распадается с периодом полураспада $t_S$. Как ни странно, он с равной вероятностью распадается либо на некий изотоп Мидия, $Mi$, либо на малоизвестный стабильный элемент Эндий. Мидий также радиоактивен и распадается с периодом полураспада.
$t_M$ в Эндий. Все $St$ было в звездном веществе, сконденсировавшемся на планете, а все $Mi$ и $En$ возникли в результате описанного пути распада. Кроме того, $t_M \neq t_S$.
Пусть $x(t), y(t)$ и $z(t)$ количества $St$, $Mi$ и $En$ на планете так, что $x(0) = 1$.
\begin{itemize}
    \item Оцените $x, y, z$ в зависимости от $t$. Какие ограничения
значения при $t \to \infty$?
    \item Решите дифференциальные уравнения для $x, y$ и $z$. Пусть $\sigma$ -- константа затухания для $St$ и $\mu$ для Mi. Hint: сумма $x+y+z$ постоянна, поэтому мы должны иметь $\dot{x}+\dot{y} +\dot{z} = 0$.
    \item В какое время количество Мидиума достигает пика?
    \item Предположим, что вместо $x(0) = 1$ у нас было $x(0) = 2$. Какое изменение это внесет в $x(t), y(t)$ и $z(t)$?
    \item Несвязанный вопрос: предположим, что $x(t) = e^t$ является решением дифференциального уравнения $t\dot{x} + 2x = q(t)$. Найти $q(t)$. Каково общее решение?
\end{itemize} 

\end{task}

\begin{task}
Решите уравнение $y'+2y=y^2e^x$
\end{task}

\begin{task}
Решите уравнение $xy^2y'=x^2+y^2$
\end{task}

\begin{task}
Решите уравнение $y'x^3\sin y = xy'-2y$
\end{task}

\begin{task}
Решите уравнение сведением к линейному $xdx=(x^2-2y+1)dy$
\end{task}

\begin{task}
Решите уравнение сведением к линейному $x(e^y-y')=2$
\end{task}

\begin{task}
Решите уравнение сведением к линейному $y(x)=\int_0^xy(t)dt+x+1$
\end{task}

\begin{task}
Найти траектории ортогональные к линиям семейства $y^2=Ce^x+x+1$
\end{task}

\begin{task}
Найти решение уравнения $y'\sin2x=2(y+\cos x)$, которое является ограниченным при $x \to \pi/2$
\end{task}

\begin{task}
Найти периодическое решение уравнения $y'=2y\cos^2 x-\sin x$
\end{task}
