\begin{task}
\textbf{Радиоактивный распад.} Требуется найти закон распада $m = m(t)$, если известна начальная масса радия $m_0, m(0)=m_0$. Скорость распада пропорциональна наличному количеству радия.
\end{task}


\begin{task}
\textbf{Давление воздуха.} Допустив, что в вертикальном воздушном столбе давление на каждом уровне обусловлено давлением вышележащих слоев, найти зависимость давления $p$ от высоты $h$, если известно, что на уровне моря $(h=0)$ это давление равно 1 кг/см$^2$, а на высоте 500м оно равно $0.92$ кг/см$^2$. Подсчитать изменение давления воздуха при переходе от слоя на высоте $h$  к слою на $h+dh$ и воспользоваться законом Бойля-Мариотта, т.е. считать, что плотность воздуха $\rho$ пропорциональна давлению $p$.
\end{task}

\begin{task}
    \textbf{Вода в полусфере.} За какое время вода, заполняющая полусферическую чашу диаметром 2 м, вытечет из неё через круглое отверстие диаметром $0.2$ м, вырезанное в дне чаши, если скорость вытекания воды $v=0.6\sqrt{2gh}$ см/c, где $h$ -- высота столба воды над отверстием?
\end{task}

\begin{task}
   Найти кривую, для которой треугольник, образованный нормалью с осями координат, был бы равновелик треугольнику, образованному осью $x$, касательной и нормалью.
\end{task}

\begin{task}
    \textbf{Вымывание соли.} В баке находится 100 л , содержащего 10 кг соли. В бак непрерывно подаётся вода (5 л/мин), которая перемешивается с имеющимся раствором. Смесь вытекает с той же скоростью. Сколько соли в баке останется через час?
\end{task}  


\begin{task}
Найти кривые, обладающие следующим свойством: отрезок оси абсцисс, отсекаемый касательной и нормалью, проведенными из произвольной точки кривой, равен $2a$.
\end{task}

\begin{task}
Найти кривые, у которых точка пересечения любой касательной с осью абсцисс имеет абсциссу, вдвое меньшую абсциссы точки касания.
\end{task}

\begin{task}
Доказать, что если $x(t)$ -- решение автономного уравнения $\frac{dx}{dt}=f(x)$, то $x(t+\tau), \tau \in R$ тоже решение.
\end{task}

\begin{task}
Свести в общем виде уравнение $y'=f(ax+by)$ к уравнению с разделяющимися переменными заменой $z=ax+by$ (что то же самое $z=ax+by+c$ почему?).
\end{task}
